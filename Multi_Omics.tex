% Options for packages loaded elsewhere
\PassOptionsToPackage{unicode}{hyperref}
\PassOptionsToPackage{hyphens}{url}
%
\documentclass[
]{article}
\usepackage{amsmath,amssymb}
\usepackage{iftex}
\ifPDFTeX
  \usepackage[T1]{fontenc}
  \usepackage[utf8]{inputenc}
  \usepackage{textcomp} % provide euro and other symbols
\else % if luatex or xetex
  \usepackage{unicode-math} % this also loads fontspec
  \defaultfontfeatures{Scale=MatchLowercase}
  \defaultfontfeatures[\rmfamily]{Ligatures=TeX,Scale=1}
\fi
\usepackage{lmodern}
\ifPDFTeX\else
  % xetex/luatex font selection
\fi
% Use upquote if available, for straight quotes in verbatim environments
\IfFileExists{upquote.sty}{\usepackage{upquote}}{}
\IfFileExists{microtype.sty}{% use microtype if available
  \usepackage[]{microtype}
  \UseMicrotypeSet[protrusion]{basicmath} % disable protrusion for tt fonts
}{}
\makeatletter
\@ifundefined{KOMAClassName}{% if non-KOMA class
  \IfFileExists{parskip.sty}{%
    \usepackage{parskip}
  }{% else
    \setlength{\parindent}{0pt}
    \setlength{\parskip}{6pt plus 2pt minus 1pt}}
}{% if KOMA class
  \KOMAoptions{parskip=half}}
\makeatother
\usepackage{xcolor}
\usepackage[margin=1in]{geometry}
\usepackage{color}
\usepackage{fancyvrb}
\newcommand{\VerbBar}{|}
\newcommand{\VERB}{\Verb[commandchars=\\\{\}]}
\DefineVerbatimEnvironment{Highlighting}{Verbatim}{commandchars=\\\{\}}
% Add ',fontsize=\small' for more characters per line
\usepackage{framed}
\definecolor{shadecolor}{RGB}{248,248,248}
\newenvironment{Shaded}{\begin{snugshade}}{\end{snugshade}}
\newcommand{\AlertTok}[1]{\textcolor[rgb]{0.94,0.16,0.16}{#1}}
\newcommand{\AnnotationTok}[1]{\textcolor[rgb]{0.56,0.35,0.01}{\textbf{\textit{#1}}}}
\newcommand{\AttributeTok}[1]{\textcolor[rgb]{0.13,0.29,0.53}{#1}}
\newcommand{\BaseNTok}[1]{\textcolor[rgb]{0.00,0.00,0.81}{#1}}
\newcommand{\BuiltInTok}[1]{#1}
\newcommand{\CharTok}[1]{\textcolor[rgb]{0.31,0.60,0.02}{#1}}
\newcommand{\CommentTok}[1]{\textcolor[rgb]{0.56,0.35,0.01}{\textit{#1}}}
\newcommand{\CommentVarTok}[1]{\textcolor[rgb]{0.56,0.35,0.01}{\textbf{\textit{#1}}}}
\newcommand{\ConstantTok}[1]{\textcolor[rgb]{0.56,0.35,0.01}{#1}}
\newcommand{\ControlFlowTok}[1]{\textcolor[rgb]{0.13,0.29,0.53}{\textbf{#1}}}
\newcommand{\DataTypeTok}[1]{\textcolor[rgb]{0.13,0.29,0.53}{#1}}
\newcommand{\DecValTok}[1]{\textcolor[rgb]{0.00,0.00,0.81}{#1}}
\newcommand{\DocumentationTok}[1]{\textcolor[rgb]{0.56,0.35,0.01}{\textbf{\textit{#1}}}}
\newcommand{\ErrorTok}[1]{\textcolor[rgb]{0.64,0.00,0.00}{\textbf{#1}}}
\newcommand{\ExtensionTok}[1]{#1}
\newcommand{\FloatTok}[1]{\textcolor[rgb]{0.00,0.00,0.81}{#1}}
\newcommand{\FunctionTok}[1]{\textcolor[rgb]{0.13,0.29,0.53}{\textbf{#1}}}
\newcommand{\ImportTok}[1]{#1}
\newcommand{\InformationTok}[1]{\textcolor[rgb]{0.56,0.35,0.01}{\textbf{\textit{#1}}}}
\newcommand{\KeywordTok}[1]{\textcolor[rgb]{0.13,0.29,0.53}{\textbf{#1}}}
\newcommand{\NormalTok}[1]{#1}
\newcommand{\OperatorTok}[1]{\textcolor[rgb]{0.81,0.36,0.00}{\textbf{#1}}}
\newcommand{\OtherTok}[1]{\textcolor[rgb]{0.56,0.35,0.01}{#1}}
\newcommand{\PreprocessorTok}[1]{\textcolor[rgb]{0.56,0.35,0.01}{\textit{#1}}}
\newcommand{\RegionMarkerTok}[1]{#1}
\newcommand{\SpecialCharTok}[1]{\textcolor[rgb]{0.81,0.36,0.00}{\textbf{#1}}}
\newcommand{\SpecialStringTok}[1]{\textcolor[rgb]{0.31,0.60,0.02}{#1}}
\newcommand{\StringTok}[1]{\textcolor[rgb]{0.31,0.60,0.02}{#1}}
\newcommand{\VariableTok}[1]{\textcolor[rgb]{0.00,0.00,0.00}{#1}}
\newcommand{\VerbatimStringTok}[1]{\textcolor[rgb]{0.31,0.60,0.02}{#1}}
\newcommand{\WarningTok}[1]{\textcolor[rgb]{0.56,0.35,0.01}{\textbf{\textit{#1}}}}
\usepackage{graphicx}
\makeatletter
\def\maxwidth{\ifdim\Gin@nat@width>\linewidth\linewidth\else\Gin@nat@width\fi}
\def\maxheight{\ifdim\Gin@nat@height>\textheight\textheight\else\Gin@nat@height\fi}
\makeatother
% Scale images if necessary, so that they will not overflow the page
% margins by default, and it is still possible to overwrite the defaults
% using explicit options in \includegraphics[width, height, ...]{}
\setkeys{Gin}{width=\maxwidth,height=\maxheight,keepaspectratio}
% Set default figure placement to htbp
\makeatletter
\def\fps@figure{htbp}
\makeatother
\setlength{\emergencystretch}{3em} % prevent overfull lines
\providecommand{\tightlist}{%
  \setlength{\itemsep}{0pt}\setlength{\parskip}{0pt}}
\setcounter{secnumdepth}{-\maxdimen} % remove section numbering
\ifLuaTeX
  \usepackage{selnolig}  % disable illegal ligatures
\fi
\IfFileExists{bookmark.sty}{\usepackage{bookmark}}{\usepackage{hyperref}}
\IfFileExists{xurl.sty}{\usepackage{xurl}}{} % add URL line breaks if available
\urlstyle{same}
\hypersetup{
  pdftitle={Forth Day Multi Omics},
  pdfauthor={Mark davids},
  hidelinks,
  pdfcreator={LaTeX via pandoc}}

\title{Forth Day Multi Omics}
\author{Mark davids}
\date{15 November, 2024}

\begin{document}
\maketitle

\begin{Shaded}
\begin{Highlighting}[]
\FunctionTok{library}\NormalTok{(vegan)}
\FunctionTok{library}\NormalTok{(phyloseq)}
\FunctionTok{library}\NormalTok{(ggplot2)}
\FunctionTok{library}\NormalTok{(microbiome)}
\FunctionTok{library}\NormalTok{(ggpubr)}
\FunctionTok{library}\NormalTok{(tidyverse)}
\FunctionTok{library}\NormalTok{(mixOmics)}
\end{Highlighting}
\end{Shaded}

\hypertarget{multi-omics}{%
\section{Multi Omics}\label{multi-omics}}

Here we will explore two methods for ``multiomics'' analysis. The
dataset used are baseline samples from the MPS study, which is composed
of healthy FMT donors and the metSyn patients. The two omics datasets
are fecal microbiome composition and fasted plasma metabolomics.

During the first analysis we will test if distance matrix of each of the
two datasets have similar structures. The second analysis consists of a
Regularized Canonical Correlation Analysis with which we will try to
identify between datasets relation ships.

\hypertarget{load-and-subset-to-equal-datasets}{%
\section{Load and subset to equal
datasets}\label{load-and-subset-to-equal-datasets}}

\begin{Shaded}
\begin{Highlighting}[]
\NormalTok{MPS.metabs }\OtherTok{\textless{}{-}} \FunctionTok{readRDS}\NormalTok{(}\StringTok{"MPS.metabs.RDS"}\NormalTok{)}
\FunctionTok{taxa\_names}\NormalTok{(MPS.metabs) }\OtherTok{\textless{}{-}}\NormalTok{ MPS.metabs}\SpecialCharTok{@}\NormalTok{tax\_table[,}\StringTok{"BIOCHEMICAL"}\NormalTok{]}

\NormalTok{MPS}\FloatTok{.16}\NormalTok{S }\OtherTok{\textless{}{-}} \FunctionTok{readRDS}\NormalTok{(}\StringTok{"MPS.16S.triads.mod.RDS"}\NormalTok{)}
\NormalTok{MPS}\FloatTok{.16}\NormalTok{S }\OtherTok{\textless{}{-}} \FunctionTok{prune\_samples}\NormalTok{(MPS}\FloatTok{.16}\NormalTok{S}\SpecialCharTok{@}\NormalTok{sam\_data}\SpecialCharTok{$}\NormalTok{Sample\_Type}\SpecialCharTok{!=}\StringTok{"PostFMT"}\NormalTok{, MPS}\FloatTok{.16}\NormalTok{S)}
\NormalTok{MPS}\FloatTok{.16}\NormalTok{S}\SpecialCharTok{@}\NormalTok{sam\_data}\SpecialCharTok{$}\NormalTok{Subject\_ID }\SpecialCharTok{\%in\%} \FunctionTok{sample\_names}\NormalTok{(MPS.metabs)}
\end{Highlighting}
\end{Shaded}

\begin{verbatim}
##  [1] TRUE TRUE TRUE TRUE TRUE TRUE TRUE TRUE TRUE TRUE TRUE TRUE TRUE TRUE TRUE
## [16] TRUE TRUE TRUE TRUE TRUE TRUE TRUE TRUE TRUE TRUE TRUE TRUE TRUE TRUE TRUE
## [31] TRUE TRUE TRUE TRUE TRUE TRUE TRUE TRUE TRUE TRUE TRUE TRUE TRUE TRUE
\end{verbatim}

\begin{Shaded}
\begin{Highlighting}[]
\FunctionTok{sample\_names}\NormalTok{(MPS}\FloatTok{.16}\NormalTok{S) }\OtherTok{\textless{}{-}}\NormalTok{ MPS}\FloatTok{.16}\NormalTok{S}\SpecialCharTok{@}\NormalTok{sam\_data}\SpecialCharTok{$}\NormalTok{Subject\_ID}
\FunctionTok{taxa\_names}\NormalTok{(MPS}\FloatTok{.16}\NormalTok{S) }\OtherTok{\textless{}{-}} \FunctionTok{make.unique}\NormalTok{(}\FunctionTok{paste}\NormalTok{(MPS}\FloatTok{.16}\NormalTok{S}\SpecialCharTok{@}\NormalTok{tax\_table[,}\DecValTok{6}\NormalTok{]))}
\end{Highlighting}
\end{Shaded}

\hypertarget{reorder-phyloseq-so-they-are-in-the-same-order}{%
\section{Reorder phyloseq so they are in the same
order}\label{reorder-phyloseq-so-they-are-in-the-same-order}}

\begin{Shaded}
\begin{Highlighting}[]
\FunctionTok{sample\_names}\NormalTok{(MPS}\FloatTok{.16}\NormalTok{S) }\SpecialCharTok{==} \FunctionTok{sample\_names}\NormalTok{(MPS.metabs)}
\end{Highlighting}
\end{Shaded}

\begin{verbatim}
##  [1] FALSE FALSE FALSE FALSE FALSE FALSE FALSE FALSE FALSE FALSE FALSE FALSE
## [13] FALSE FALSE FALSE FALSE FALSE FALSE FALSE FALSE FALSE FALSE FALSE FALSE
## [25] FALSE FALSE FALSE FALSE FALSE FALSE FALSE FALSE FALSE FALSE FALSE FALSE
## [37] FALSE FALSE FALSE FALSE FALSE FALSE FALSE FALSE
\end{verbatim}

\begin{Shaded}
\begin{Highlighting}[]
\NormalTok{MPS.metabs }\OtherTok{\textless{}{-}} \FunctionTok{phyloseq}\NormalTok{(MPS.metabs}\SpecialCharTok{@}\NormalTok{otu\_table[, }\FunctionTok{sample\_names}\NormalTok{(MPS}\FloatTok{.16}\NormalTok{S)],}
                       \FunctionTok{tax\_table}\NormalTok{(MPS.metabs))}

\FunctionTok{sample\_names}\NormalTok{(MPS}\FloatTok{.16}\NormalTok{S) }\SpecialCharTok{==} \FunctionTok{sample\_names}\NormalTok{(MPS.metabs)}
\end{Highlighting}
\end{Shaded}

\begin{verbatim}
##  [1] TRUE TRUE TRUE TRUE TRUE TRUE TRUE TRUE TRUE TRUE TRUE TRUE TRUE TRUE TRUE
## [16] TRUE TRUE TRUE TRUE TRUE TRUE TRUE TRUE TRUE TRUE TRUE TRUE TRUE TRUE TRUE
## [31] TRUE TRUE TRUE TRUE TRUE TRUE TRUE TRUE TRUE TRUE TRUE TRUE TRUE TRUE
\end{verbatim}

\begin{Shaded}
\begin{Highlighting}[]
\FunctionTok{sample\_data}\NormalTok{(MPS.metabs) }\OtherTok{\textless{}{-}} \FunctionTok{sample\_data}\NormalTok{(MPS}\FloatTok{.16}\NormalTok{S)}
\end{Highlighting}
\end{Shaded}

In order to reduce dimensions a bit we will aggregate microbiome data on
Genus level. We are also not interested in Xenobiotics from the
metabolomics panel.

\begin{Shaded}
\begin{Highlighting}[]
\NormalTok{MPS}\FloatTok{.16}\NormalTok{S }\OtherTok{\textless{}{-}}\NormalTok{ microbiome}\SpecialCharTok{::}\FunctionTok{aggregate\_taxa}\NormalTok{(MPS}\FloatTok{.16}\NormalTok{S, }\StringTok{"Genus"}\NormalTok{)}
\NormalTok{MPS.metabs }\OtherTok{\textless{}{-}} \FunctionTok{prune\_taxa}\NormalTok{(}\SpecialCharTok{!}\NormalTok{MPS.metabs}\SpecialCharTok{@}\NormalTok{tax\_table[,}\StringTok{"SUPER PATHWAY"}\NormalTok{] }\SpecialCharTok{\%in\%} \StringTok{"Xenobiotics"}\NormalTok{ , MPS.metabs)}
\end{Highlighting}
\end{Shaded}

For the CCA it is better to reduce the number of features. We will
reduce it by only selecting 100 features with the highest variability.

\begin{Shaded}
\begin{Highlighting}[]
\NormalTok{nfeat}\OtherTok{=}\DecValTok{100}
\NormalTok{asvs }\OtherTok{\textless{}{-}} \FunctionTok{apply}\NormalTok{(microbiome}\SpecialCharTok{::}\FunctionTok{transform}\NormalTok{(MPS}\FloatTok{.16}\NormalTok{S, }\StringTok{"Z"}\NormalTok{)}\SpecialCharTok{@}\NormalTok{otu\_table}\SpecialCharTok{@}\NormalTok{.Data,}\DecValTok{1}\NormalTok{,}\ControlFlowTok{function}\NormalTok{(x) }\FunctionTok{sum}\NormalTok{(}\FunctionTok{abs}\NormalTok{(x))) }\SpecialCharTok{\%\textgreater{}\%} \FunctionTok{sort}\NormalTok{(}\AttributeTok{decreasing =}\NormalTok{ T) }\SpecialCharTok{\%\textgreater{}\%} \FunctionTok{head}\NormalTok{(nfeat) }\SpecialCharTok{\%\textgreater{}\%} \FunctionTok{names}\NormalTok{()}

\NormalTok{MPS}\FloatTok{.16}\NormalTok{S.reduced }\OtherTok{\textless{}{-}} \FunctionTok{prune\_taxa}\NormalTok{(asvs, MPS}\FloatTok{.16}\NormalTok{S)}

\NormalTok{metabs }\OtherTok{\textless{}{-}} \FunctionTok{apply}\NormalTok{(microbiome}\SpecialCharTok{::}\FunctionTok{transform}\NormalTok{(MPS.metabs, }\StringTok{"Z"}\NormalTok{)}\SpecialCharTok{@}\NormalTok{otu\_table}\SpecialCharTok{@}\NormalTok{.Data,}\DecValTok{1}\NormalTok{,}\ControlFlowTok{function}\NormalTok{(x) }\FunctionTok{sum}\NormalTok{(}\FunctionTok{abs}\NormalTok{(x))) }\SpecialCharTok{\%\textgreater{}\%} \FunctionTok{sort}\NormalTok{(}\AttributeTok{decreasing =}\NormalTok{ T) }\SpecialCharTok{\%\textgreater{}\%} \FunctionTok{head}\NormalTok{(nfeat) }\SpecialCharTok{\%\textgreater{}\%} \FunctionTok{names}\NormalTok{()}

\NormalTok{MPS.metabs.reduced }\OtherTok{\textless{}{-}} \FunctionTok{prune\_taxa}\NormalTok{(metabs, MPS.metabs)}
\end{Highlighting}
\end{Shaded}

\hypertarget{regular-ordination}{%
\section{Regular ordination}\label{regular-ordination}}

First we have a look at the datasets individually.

\begin{Shaded}
\begin{Highlighting}[]
\FunctionTok{plot\_ordination}\NormalTok{(}\AttributeTok{physeq =}\NormalTok{ MPS}\FloatTok{.16}\NormalTok{S.reduced, }\AttributeTok{ordination =} \FunctionTok{ordinate}\NormalTok{(MPS}\FloatTok{.16}\NormalTok{S.reduced, }\StringTok{"PCoA"}\NormalTok{, }\StringTok{"bray"}\NormalTok{), }\AttributeTok{color=}\StringTok{"Sample\_Type"}\NormalTok{) }\SpecialCharTok{+} 
  \FunctionTok{labs}\NormalTok{(}\AttributeTok{title=}\StringTok{"PCoA Microbiome"}\NormalTok{)}
\end{Highlighting}
\end{Shaded}

\includegraphics{Multi_Omics_files/figure-latex/unnamed-chunk-6-1.pdf}

\begin{Shaded}
\begin{Highlighting}[]
\FunctionTok{plot\_ordination}\NormalTok{(}\AttributeTok{physeq =}\NormalTok{ MPS.metabs.reduced, }\AttributeTok{ordination =} \FunctionTok{ordinate}\NormalTok{(MPS.metabs.reduced, }\StringTok{"PCoA"}\NormalTok{, }\StringTok{"bray"}\NormalTok{), }\AttributeTok{color=}\StringTok{"Sample\_Type"}\NormalTok{) }\SpecialCharTok{+} 
  \FunctionTok{labs}\NormalTok{(}\AttributeTok{title=}\StringTok{"PCoA Metabolites"}\NormalTok{)}
\end{Highlighting}
\end{Shaded}

\includegraphics{Multi_Omics_files/figure-latex/unnamed-chunk-6-2.pdf}

\begin{Shaded}
\begin{Highlighting}[]
\NormalTok{d }\OtherTok{\textless{}{-}}\NormalTok{ phyloseq}\SpecialCharTok{::}\FunctionTok{distance}\NormalTok{(}\AttributeTok{physeq =}\NormalTok{ MPS}\FloatTok{.16}\NormalTok{S, }\StringTok{"bray"}\NormalTok{)}
\FunctionTok{adonis2}\NormalTok{(d}\SpecialCharTok{\textasciitilde{}}\NormalTok{MPS}\FloatTok{.16}\NormalTok{S}\SpecialCharTok{@}\NormalTok{sam\_data}\SpecialCharTok{$}\NormalTok{Sample\_Type)}
\end{Highlighting}
\end{Shaded}

\begin{verbatim}
## Permutation test for adonis under reduced model
## Terms added sequentially (first to last)
## Permutation: free
## Number of permutations: 999
## 
## adonis2(formula = d ~ MPS.16S@sam_data$Sample_Type)
##                              Df SumOfSqs      R2      F Pr(>F)  
## MPS.16S@sam_data$Sample_Type  1   0.1732 0.03839 1.6768  0.062 .
## Residual                     42   4.3394 0.96161                
## Total                        43   4.5127 1.00000                
## ---
## Signif. codes:  0 '***' 0.001 '**' 0.01 '*' 0.05 '.' 0.1 ' ' 1
\end{verbatim}

\begin{Shaded}
\begin{Highlighting}[]
\NormalTok{d }\OtherTok{\textless{}{-}}\NormalTok{ phyloseq}\SpecialCharTok{::}\FunctionTok{distance}\NormalTok{(}\AttributeTok{physeq =}\NormalTok{ MPS.metabs, }\StringTok{"bray"}\NormalTok{)}
\FunctionTok{adonis2}\NormalTok{(d}\SpecialCharTok{\textasciitilde{}}\NormalTok{MPS.metabs}\SpecialCharTok{@}\NormalTok{sam\_data}\SpecialCharTok{$}\NormalTok{Sample\_Type)}
\end{Highlighting}
\end{Shaded}

\begin{verbatim}
## Permutation test for adonis under reduced model
## Terms added sequentially (first to last)
## Permutation: free
## Number of permutations: 999
## 
## adonis2(formula = d ~ MPS.metabs@sam_data$Sample_Type)
##                                 Df SumOfSqs     R2      F Pr(>F)    
## MPS.metabs@sam_data$Sample_Type  1  0.13204 0.0672 3.0257  0.001 ***
## Residual                        42  1.83292 0.9328                  
## Total                           43  1.96497 1.0000                  
## ---
## Signif. codes:  0 '***' 0.001 '**' 0.01 '*' 0.05 '.' 0.1 ' ' 1
\end{verbatim}

Not obvious from the ordination, but there is a significant difference
in plasma metabolites between healthy donors and metsyn subjects.

\hypertarget{procrustes}{%
\section{Procrustes}\label{procrustes}}

We will use procrustes to test coherence between the two datasets. In
order to run procrustes we will need ordinations of the dataset. For 16S
we use default Bray-Curtis distance. For metabolites we will use regular
euclidean distance. For both sets we will use PCoA for ordination.

To test the probability of getting a better fit, we will perform the
protest permutation test.

\begin{Shaded}
\begin{Highlighting}[]
\NormalTok{ord}\FloatTok{.16}\NormalTok{S }\OtherTok{\textless{}{-}} \FunctionTok{ordinate}\NormalTok{(MPS}\FloatTok{.16}\NormalTok{S, }\StringTok{"PCoA"}\NormalTok{, }\StringTok{"bray"}\NormalTok{)}
\CommentTok{\#ord.metabs \textless{}{-} ordinate(microbiome::transform(MPS.metabs, "clr"), "PCoA", "euclidean")}
\NormalTok{ord.metabs }\OtherTok{\textless{}{-}} \FunctionTok{ordinate}\NormalTok{(MPS.metabs, }\StringTok{"PCoA"}\NormalTok{, }\StringTok{"euclidean"}\NormalTok{)}

\FunctionTok{plot}\NormalTok{(}\FunctionTok{procrustes}\NormalTok{(}\AttributeTok{X =}\NormalTok{ ord}\FloatTok{.16}\NormalTok{S}\SpecialCharTok{$}\NormalTok{vectors, }\AttributeTok{Y =}\NormalTok{ ord.metabs}\SpecialCharTok{$}\NormalTok{vectors, }\AttributeTok{scale=}\NormalTok{T))}
\end{Highlighting}
\end{Shaded}

\includegraphics{Multi_Omics_files/figure-latex/unnamed-chunk-8-1.pdf}

\begin{Shaded}
\begin{Highlighting}[]
\FunctionTok{protest}\NormalTok{(}\AttributeTok{X =}\NormalTok{ ord}\FloatTok{.16}\NormalTok{S}\SpecialCharTok{$}\NormalTok{vectors, }\AttributeTok{Y =}\NormalTok{ ord.metabs}\SpecialCharTok{$}\NormalTok{vectors, }\AttributeTok{permutations =} \DecValTok{99999}\NormalTok{)}
\end{Highlighting}
\end{Shaded}

\begin{verbatim}
## 
## Call:
## protest(X = ord.16S$vectors, Y = ord.metabs$vectors, permutations = 99999) 
## 
## Procrustes Sum of Squares (m12 squared):        0.5193 
## Correlation in a symmetric Procrustes rotation: 0.6933 
## Significance:  0.00016 
## 
## Permutation: free
## Number of permutations: 99999
\end{verbatim}

These results show a significant relation between microbiome composition
and circulating metabolites.

\hypertarget{rcca}{%
\section{rCCA}\label{rcca}}

CCA is dimensionality reduction approach by maximizing the correlation
of the features between the datasets. Variance in the data that is not
coherent between both datasets is thereby reduced.

We will apply a regularized form of CCA, which is required if you want
to perform CCA on datasets with a higher number of features than
samples.

\begin{Shaded}
\begin{Highlighting}[]
\NormalTok{X }\OtherTok{\textless{}{-}} \FunctionTok{t}\NormalTok{(MPS}\FloatTok{.16}\NormalTok{S.reduced}\SpecialCharTok{@}\NormalTok{otu\_table}\SpecialCharTok{@}\NormalTok{.Data)}
\NormalTok{Y }\OtherTok{\textless{}{-}} \FunctionTok{t}\NormalTok{(MPS.metabs.reduced}\SpecialCharTok{@}\NormalTok{otu\_table}\SpecialCharTok{@}\NormalTok{.Data)}
\end{Highlighting}
\end{Shaded}

\hypertarget{perform-the-cca}{%
\section{perform the CCA}\label{perform-the-cca}}

There are two regularisation methods available. Here we apply the
shrinkage method as feature counts are substantially larger than the
number of samples.

\begin{Shaded}
\begin{Highlighting}[]
\NormalTok{rcc.MPS }\OtherTok{\textless{}{-}} \FunctionTok{rcc}\NormalTok{(X,Y, }\AttributeTok{method =} \StringTok{\textquotesingle{}shrinkage\textquotesingle{}}\NormalTok{) }
\CommentTok{\# examine the optimal lambda values after shrinkage }
\FunctionTok{plot}\NormalTok{(rcc.MPS, }\AttributeTok{type =} \StringTok{"barplot"}\NormalTok{, }\AttributeTok{main =} \StringTok{"Shrinkage"}\NormalTok{) }
\end{Highlighting}
\end{Shaded}

\includegraphics{Multi_Omics_files/figure-latex/unnamed-chunk-10-1.pdf}

The scree plot shows that there is a small drop off after two components
indicating these two components capture more variance than by chance.

We can plot the samples in this CCA ordination.

\begin{Shaded}
\begin{Highlighting}[]
\FunctionTok{plotIndiv}\NormalTok{(rcc.MPS, }\AttributeTok{comp =} \DecValTok{1}\SpecialCharTok{:}\DecValTok{2}\NormalTok{, }
          \AttributeTok{ind.names =} \ConstantTok{NULL}\NormalTok{,}
          \AttributeTok{group =}\NormalTok{ MPS}\FloatTok{.16}\NormalTok{S}\SpecialCharTok{@}\NormalTok{sam\_data}\SpecialCharTok{$}\NormalTok{Sample\_Type, }\AttributeTok{rep.space =} \StringTok{"XY{-}variate"}\NormalTok{, }
          \AttributeTok{legend =} \ConstantTok{TRUE}\NormalTok{, }\AttributeTok{title =} \StringTok{\textquotesingle{}MPS, rCCA shrinkage XY{-}space\textquotesingle{}}\NormalTok{)}
\end{Highlighting}
\end{Shaded}

\includegraphics{Multi_Omics_files/figure-latex/unnamed-chunk-11-1.pdf}

This plot shows that if we focus only on coherent variance between the
dataset, out healthy donor samples separate from preFMT samples.\\
This implies there is a relation between fecal microbiome, plasma
metabolites and health status (or other confounding factor)

We can plot the projection of both datasets, similar to the procrustus
plot.

\begin{Shaded}
\begin{Highlighting}[]
\FunctionTok{plotArrow}\NormalTok{(rcc.MPS, }\AttributeTok{group =}\NormalTok{ MPS}\FloatTok{.16}\NormalTok{S}\SpecialCharTok{@}\NormalTok{sam\_data}\SpecialCharTok{$}\NormalTok{Sample\_Type, }
          \AttributeTok{col.per.group =} \FunctionTok{color.mixo}\NormalTok{(}\DecValTok{1}\SpecialCharTok{:}\DecValTok{5}\NormalTok{),}
          \AttributeTok{title =} \StringTok{\textquotesingle{}(b) Nutrimouse, shrinkage method\textquotesingle{}}\NormalTok{)}
\end{Highlighting}
\end{Shaded}

\includegraphics{Multi_Omics_files/figure-latex/unnamed-chunk-12-1.pdf}

Besides the relations between the samples, we can also use CCA to find
correlations between the datasets from the feature perspective. We can
use either a circle plot, or a heatmap to visualize the correlations
between the feature sets.

\begin{Shaded}
\begin{Highlighting}[]
\FunctionTok{plotVar}\NormalTok{(rcc.MPS, }\AttributeTok{var.names =} \FunctionTok{c}\NormalTok{(}\ConstantTok{TRUE}\NormalTok{, }\ConstantTok{TRUE}\NormalTok{),}
        \AttributeTok{cex =} \FunctionTok{c}\NormalTok{(}\DecValTok{3}\NormalTok{, }\DecValTok{3}\NormalTok{), }\AttributeTok{cutoff =} \FloatTok{0.5}\NormalTok{,}
        \AttributeTok{title =} \StringTok{\textquotesingle{}(b) Nutrimouse, rCCA shrinkage comp 1 {-} 2\textquotesingle{}}\NormalTok{)}
\end{Highlighting}
\end{Shaded}

\includegraphics{Multi_Omics_files/figure-latex/unnamed-chunk-13-1.pdf}

\begin{Shaded}
\begin{Highlighting}[]
\FunctionTok{cim}\NormalTok{(rcc.MPS, }\AttributeTok{comp =} \DecValTok{1}\SpecialCharTok{:}\DecValTok{2}\NormalTok{, }\AttributeTok{xlab =} \StringTok{"metabolites"}\NormalTok{, }\AttributeTok{ylab =} \StringTok{"taxa"}\NormalTok{)}
\end{Highlighting}
\end{Shaded}

A difference between groups was not obvious from the unconstrained
ordinations but now using a constrained approach shows there are
interactions between omics datasets that seperate our groups. Some
correlations between metabolites and microbes have also been identified,
through the correlations are not very strong.

\end{document}
